\documentclass[10pt]{beamer}
\usetheme{Boadilla}

\usepackage{hyperref}
\usepackage{graphicx}
\usepackage{subfig}
\usepackage{amsmath,amssymb}

\graphicspath{ {images/} }

\usepackage{tikz}

%Some useful commands for QM
\newcommand{\bra}[1]{\left< #1 \right|}
\newcommand{\ket}[1]{\left| #1 \right>}
\newcommand{\expVal}[1]{\left< #1 \right>}
\newcommand{\braket}[2]{\left<#1|#2\right>}

\title{Purification Complexity of Gaussian States}
\subtitle{arxiv:181x.xxxxx, work in progress with Elena C\'aceres, Shira Chapman, Juan Pablo Hernandez, Rob Myers, and Shan-Ming Ruan}
\author{Josiah Couch}
\institute{University of Texas at Austin}
\date{20 Oct 2018}



\begin{document}

\begin{frame}
\titlepage\end{frame}

\begin{frame}
\frametitle{The AdS/CFT correspondence}

\end{frame}

\begin{frame}
\frametitle{The holgraphic complexity conjecture}

\end{frame}

\begin{frame}
\frametitle{Complexity in field theory?}

\end{frame}

\begin{frame}
\frametitle{Subregion complexity}

\end{frame}

\begin{frame}
\frametitle{Purification complexity}

\end{frame}

\begin{frame}
\frametitle{Purification complexity in field theory}

\end{frame}

\begin{frame}
\frametitle{Results}

\end{frame}

\end{document}
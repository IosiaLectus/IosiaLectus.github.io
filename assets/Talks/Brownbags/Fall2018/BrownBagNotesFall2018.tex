\documentclass[12pt]{amsart}

\usepackage[caption = false]{subfig}
\usepackage[normalem]{ulem}

%Some useful commands for QM
\newcommand{\bra}[1]{\left< #1 \right|}
\newcommand{\ket}[1]{\left| #1 \right>}
\newcommand{\expVal}[1]{\left< #1 \right>}
\newcommand{\braket}[2]{\left<#1|#2\right>}

\usepackage{amsthm,enumitem,graphicx,hyperref,color}
\graphicspath{{./img/}}
\linespread{1.6}

\author{Josiah Couch}

\date{11 October 2018}

\begin{document}

\title{Purification Complexity in Quantum Mechanics}

\maketitle

\centerline{arxiv:181x.xxxxx}

\centerline{Work in progress with Elena C\'aceres, Shira Chapman, Juan Hernandez, Rob Myers, and Shan-Ming Ruan}

\tableofcontents

\section{Black Holes and Complexity}

One of the major open problems in AdS/CFT is what in the CFT is dual to the behind the horizon geometry of a black hole solution. There is some sense in which the answer cannot be any of the usual quantities we consider in on a CFT (or rather a double copy of a CFT on which we may purify a thermal state), because the behind the horizon geometry is dynamic (in that there is not timelike killing vector), and yet the dual state is thermal. The usual intuition is that thermal states are not dynamic on time scales larger than a thermal time. Susskind has conjectured that certain quantities sensitive to the behind the horizon  region, namely the volume of maximal spatial slices, or the action evaluated on the causal development of such slices \footnote{With reflecting boudary conditions, a cauchy slice strictly speaking has a domain of dependence consisting of the whole space time. However, if we cut off a cauchy slice at some finite value of the bulk coordinate, then the regulated WDW patch is the domain of dependence of this regulated slice, and we recover the full WDW patch as this regulator approaches the asymptotic boundary.}(The Wheeler-DeWitt(WDW) patch) are dual to the circuit complexity of the dual quantum state. The evidence for this conjecture is somewhat circumstantial, and was reviewed recently in the talk by Stefan. Here, I will just give a brief review of the main points: 

\begin{itemize}

\item With a proper choice of normalization constant, the holographic complexity grows like temperature times entropy for black holes. This is the expected behavior for a fast scrambling system based on circuit arguments. 

\item Holographic complexity in black holes displays the 'switchback effect', as expected, once again based on circuit arguments. 

\item The holographic complexity grows in time without bound, and grows linearly at late time. Because most states are nearly maximally complex, there are statistical arguments that the complexity grows with overwhelming probabillity when starting with a low complexity state There are some (far from rigorous) arguments that the complexity should even grow linearly.

\item Of course in quantum mechanics, the complexity saturates to a maximum, wheras holographically the action/volume continue growing forever. This is perhaps to be expected though, as the holographic complexity is at best valid to $\frac{1}{N}$, and the maximum complexity grows without bound as we increase the dimension of the Hilbert space. 

\end{itemize}

Of course, these points form far from an airtight case that complexity  is dual to either the action evaluated on a WDW patch or the volume of a maximal spatial slice, and they don't distinguish between the two possibillities (or necessarily between these two and others). As such, furthere tests are warranted before the conjecture is accepted.

The most straightforward test of this conjecture would be to compute the complexity of e.g. the ground state, or perhaps the thermofield double state, of $\mathcal{N} = 4$ SYM, and compare the result to explicit holographic computations. While this computation has not been attempted, and is mostly likely extremely difficult, some recent effort, particularly by Rob Myers and collaborators, has been devoted to underastanding circuit complexity in regulated field theories. 

\section{Circuit Complexity Review}

Before we get into circuit complexity in field theory, let me first remind you of the definition of circuit complexity for unitaries and for states: Choose a tolerance $\epsilon$ and a set of unitaries $\mathcal{G} = \{g_1, g_2, ...., g_N\}$ such that the group $\expVal{\mathcal{G}}$ generated by $\mathcal{G}$ has at least one element in any $\epsilon$-ball subset of the unitary group. Then a presentation $Q$ of an element of of $\expVal{\mathcal{G}}$ can be called a quantum circuit, and we define the {\it complexity} of $Q$, denoted by $C(Q)$, to be the number of generators in the presentation. I will often abuse notation by using $Q$ interchangably for an element of $\expVal{\mathcal{G}}$ and a particular presentation of that element. We may then define the complexity of a unitary operator $U$ to be the minimal complexity over all circuits which sit inside (or rather present groupe element that sit inside) of an $\epsilon$-ball around $U$.

Note that though we may be tempted to call $\mathcal{G}$ the gateset with which we build circuits in $\expVal{\mathcal{G}}$, this does not quite match the usual notion of a gate set in computer science. To recover the usual definition, we must first choose a canonical factorization of our total Hilbert space as $\mathcal{H}_{\text{tot}} = \mathcal{H}^{\otimes N}$. Then a gate in the usual sense would be an operator defined on some small number $k < N$ of copies of $\mathcal{H}$, and the elements of $\mathcal{G}$ would be obtained as the unitaries which act as the gate on any $k$ copies of $H$ (in any order, as the gate need not be sysmetric under exchange of factors), and the identity on the rest of $\mathcal{H}_{\text{tot}}$. This allows us to extend the Hilbert space by adding additional factors of $\mathcal{H}$ while using the same set of gates (which will result in a larger set $\mathcal{G}$ on the larger Hilbert space).

Given a definition of complexity on the unitaries, we may define the relative state complexty between states $\ket{\psi}$ and state $\ket{\phi}$ as the minimum complexity over all unitaries $U$ such that $\ket{\psi} = U \ket{\phi}$. If we fix one particular choice of $\ket{\phi}$ and call this the {\it reference state}, then we may call the relative complexity betwween some other state $\ket{\psi}$ and $\ket{\phi}$ simply the circuit complexity of $\ket{\psi}$. We may also define the relative circuit complexity between unitaries $U$ and $V$ as the complexity of $UV^{\dagger}$. Under this definition, the usual notion of the complexity of $U$ is simply the relative complexity between $U$ and the identity.

Notice now that relative complexity of either unitaries or states now provides a good metric (in the sense of a distance function, not of a metric tensor) on either $\mathbb{C}P^D$ or $SU(D)$, where $D = |\mathcal{H}|$. This can be easily seen:

\begin{itemize}

\item the relative complexity $C_{\text{rel}}(U,V)$ between $U$ and $V$ manifestly non-negative, and it is zero only when $U$ is within an $\epsilon$-ball of $V$. (Okay, so it's not {\it quite} a good metric).

\item Because in the definition of complexity given above, $U$ has the same complexity as $U^{-1}$, it follows that $UV^{\dagger}$ has the same complexity as $VU^{\dagger}$, and so the relative complexity is symmetric.

\item The triangle inequality holds, because given a circuit $Q_1$ implementing $UV^{\dagger}$, and a cirucit $Q_2$ implementing $VW^{\dagger}$, $Q = Q_1Q_2$ provides a circuit implementing $UW^{\dagger}$. Because the complexity of an operator is defined as a minimum over circuits, we thus have that $C_{\text{rel}}(U,W) \leq C_{\text{rel}}(U,V) + C_{\text{rel}}(V,W)$

\end{itemize}

Similar argumetns show the relative complexity to form a metric on states. Since the complexity forms a metric, it would be nice to geometrize the complexity. One can see a geometric interpretation should hold approximately when the tolerance and the distance of the memebers of $\mathcal{G}$ (under the usual left and right invariant metric) from the identity both are taken to be extremely small. Following this idea, we may construct the {\it complexity geometry} by trading $\mathcal{G}$ for the generators (in the Lie group sense) of its elements, so that $g_i = \exp(X_i)$. Then we may write any operator as 

$$U = \exp\left( \int_0^1 ds Y^I(s) X_I \right)$$

and define the geometrized complexity of $U$ to be given by

$$F_1(Y) = \int_0^1 \sum_I \left|Y^I(s)\right| ds$$

evaluated for the $Y$ which minimizes this functional (among those which construct $U$). This provides a {\it Finsler} metric on $SU(D)$, but not a Riemannian metric. In order to have a Riemannian metric (which is easier to work with), we may define complexity with the alternative functional 

$$F_2(Y) = \int_0^1 \sqrt{G_{IJ} Y^I(s) Y^J(s)}$$

for some choice of a metric tensor $G_{IJ}$. if $G_{IJ}$ has a negative curvature, then on scales much larger than this curvature, these two functionals can be made approximately equal (This is due to the negative curvature version of the Pythagorean theorem) (explain this more clearly).

\section{Complexity in lattice regulated free scalar field theory}

Armed with a definition of complexity, we are now equipped to try to compute the complexity of some state on a quantum field theory. As typical physicists, we will start by considering harmonic oscillators, and to start with we will only consider the ground state. Here we will follow the work of (cite Jefferson and Myers here) ...

\section{Comparison to Holographic Results}

\section{Purification Complexity}

One feature of the holographic complexity examples is that they may in principle be applied to any asymptotically AdS geometry, even those that are geodesically incomplete. While most such geometries are probably not holographic, is has been argued that geometries which come as the entanglment wedge of the boundary region of a geodesically complete spacetime are dual to the reduced state on that subregion. As such, it is reasonable to think that if the one of the holographic complexity conjectures is true, then when applied to an entanglment wedge they should give the complexity of this reduced state. Here, however, we run into the problem that there is not a unique way to extend the definition of complexity outlined above to include complexities for mixed states.

\section{Purification Complexity in free scalar field theory}

\section{Comparison to subregion complexity}

%\bibliographystyle{JHEP}
%\bibliography{complexity.bib} %For bibtex

\nocite{*}

\end{document}
